%===================================== CHAP 8 =================================

\chapter{Project evaluation}
\label{ch:project_evaluation}

\section{Research phase}
\label{sec:project_evaluation-research_phase}

After the initial evaluation of the task and its requirements, it seemed more difficult than first assumed. As none of the group members had any experience writing this kind of software, a lot of time had to be dedicated to research. Building software with this size and complexity, was unlike anything taught in previous courses. Additionally, there were hundreds of pages of protocol documentation and standards, which required a lot of time to understand. All of this obviously took time, also during the development part of the project. In total, research accounted for a large portion of the time used in the project.

\subsection{Apollo}
\label{subsec:project_evaluation-research_phase-apollo}

Two full weeks were dedicated to research Apollo. As the total time of the project was less than one semester, that was a fairly large portion of it. The positive aspect of it, was that a lot was learned from this work, and some of the aspects and good practices were carried over to the implementation.

The decision to abandon Apollo was not made until the very end. It was only then that a component of the Apollo system proved to be so complex that it failed the gate criteria. After further discussion, additional issues appeared which led to the final decision. Had the research phase been shortened, it might have resulted in the decision to further develop Apollo, possibly resulting in a failure to deliver a working product. Thus, abandoning Apollo was probably the best choice. Additionally, 


\subsection{Process Model}
\label{subsec:project_evaluation-research_phase-process_model}

Although none of the group members had any experience with the phase-gate model, the choice worked out fairly well. Mainly because the first occurrence of an issue or problem that failed to meet the gate criteria would terminate further research. Subsequently, a deadline for a final decision was set, providing a firm point in time to be used in the overall project plan. This was key, as extending the research further, could potentially have had a negative impact on the development part of the project. Additionally, splitting the research work into different areas, proved effective. It caused each member to have deep insight into one area, instead of all members having some insight into everything. The negative aspect of this is that it was challenging to perform technical discussions with other group members.

The parallel research approach required frequent updates internally. Since each group member had a particular area of research, structured communication was key to keep the group sufficiently updated on current progress. The group formed what the Phase-Gate model describes as a steering-committee, that would make the final decision on whether or not to proceed. 

%During the early phases, a lot of knowledge were gained in the field of research.  Theprocess model chosen, required a lot of communication in order to succeed. As the differ-ent research tasks were separated, everyone had to contribute and explain their progressevery day.  This was challenging, as the other group members had little insight into theprogress  on  the  other  tasks.   For  this  reason,  presenting  arguments  in  a  structured  andunderstandable way was essential, and what was learned the most from.  It also requiredmore structured work than usual, especially when attempting to understand Apollo. As thesystem was so complex, figuring out one thing at a time proved to be the key to success.As mentioned, it was hard to get any help from the other group members. This causedfrustration, and was definitely one of the negative aspects. This could have been changedby having more group members devoted to one task at a time.  Instead of doing all thetasks at once, the work could have been split over two iterations.  This could have beendone by first getting an overview of the structure and learn Scala. After that, the possibilityof adding additional modules, as well as estimate time could have been done. This wouldhave made the work structured more like this.

%It is difficult to evaluate whether or not this would have been a better solution. At the one hand, it is obviously hard to work alone or in pairs on this kind of work. On the other hand, it may be more effective if you don't get completely stuck.

\subsection{Evaluation}
\label{subsec:project_evaluation-research_phase-evaluation}

The outcome of the research phase was not as exactly as hoped. The ideal path would have been to be able to proceed with Apollo. However, the knowledge the group gained from Apollo was quite extensive, and will be useful for the customer in further research and development. The research also helped in structuring the implementation. Although a lot of time was used on Apollo, researching existing solutions is an important aspect of any development project. It was also the part of the project in which the members learned the most. Not just about Apollo and doing research work, but it was a new aspect of team work. All in all, the research work was important in order to develop a best possible solution.

Looking back at the effectiveness of the model, it proved applicable for this kind of work. In a theoretical situation, where problems arose at an earlier point in time, the true effectiveness of the model would have been demonstrated. However, as previously stated, this did not occur until the end of the research phase.

In retrospective, the research period should have been compressed. By starting the research period earlier, and by working more during the period, it might have been possible to cut one week of the research period. This is the most important aspect that could have been done differently. It would have led to more time available for development.

\section{Development}
\label{sec:project_evaluation-development}

Due to the research work, the development started late. This was obviously a problem, as less of the total time could be allocated to development. When constructing the development plan, the group realized that in order to complete the project, it was necessary to develop until late may. This was obviously a risk, as there would be no time to spare if something were to go wrong towards the end. Some of the less important protocols were also removed as requirements, in order to have the time to implement the most important parts.

\subsection{Process model}
\label{subsec:project_evaluation-development-process_model}

Initially, it was decided to use test-driven development. This worked out well on the parts of the system which had a clearly defined behaviour. In that regard, the model proved successful as it made the actual code writing simpler. It was however, difficult to enforce on other parts of the system. In the end, TDD weren't a huge success, as time were spent planning how to write tests, instead of using that time to develop. 

Scrum worked out well. The most important part of it was the meetings, as they helped in keeping track of progress at all times. Additionally, any problems regarding development were discussed, and assessed by the group as a whole.

\subsection{Resource management}
\label{subsec:project_evaluation-development-time_management}

Time-estimation was one of the most difficult tasks of the project. This was partially due to the uncertainty around how much time it would take to implement protocols. The task proved way more time-consuming than the initial expectations. Additionally, the amount of research required to understand the protocols, and how to implement them were more than what was estimated.

During the development period, less time were used on development than what was planned. First of all, other courses took up a lot of time, and was not properly planned for. This could have been considered in a better way. The group weren't able to properly structure the work days, and thus tasks in other courses hindered development at times. Another factor were short term absence of group members. Although these factors were accounted for in the risk analysis, the issues that occurred weren't properly handled. The workload should have been distributed, and extra work hours introduced. However, it wasn't done in the majority of cases. The consequence to this was that a lot more time had to be used towards the end.

Combining report writing and development proved to be an issue. Although all of the group members contributed to both, it was difficult to keep track of both at the same time. That led to the person responsible for report and documentation doing a lot of work on the report. While this worked out well for the process parts of the report, he did not have as much insight into the development. Thus, it was difficult to write about the actual product. Although distributing the different parts of the report was done eventually, it should have been done earlier. This would have led to steadier progress on the report.

%\subsection{Progress}
%\label{subsec:project_evaluation-development-progress}

%During sprint three and four, the group were no longer able to keep the work flow as steady. Implementing WSN was time-consuming, and the group put too little effort into the project. A lot of time were used on research, and implementation was slower than optimal. Cutting MQTT as a requirement, led to a change in the work plan. The problem with this, was that the planned implementation was suddenly smaller than the initial hopes of both the group and customer. There was also a small misunderstanding from the requirements engineering phase. The group had interpreted that MQTT was more important than AMQP, which led to some time being used on research on MQTT.

%The group had also worked way to little, especially during sprint 4. Although other courses and research took up a lot of time, there was no real excuse for the work ethic at that point. The problem with group members being away, could also have been solved by distributing the work better. This was definitely a harsh lesson learned, and caused the group to plan a lot more hours for the remainder of the project. The development plan also had to be changed for the remaining sprints.

\subsection{Evaluation}
\label{subsec:project_evaluation-development-evaluation}

Overall, the development period worked out fairly well. The main issues were with resource and time management. Not enough work was devoted to the project in the early stages of the process. This proved to be an issue later on, especially due to the task being more difficult than the initial assessment. Although this was sub-optimal, adding more work hours was a good way to handle it.

\section{Product evaluation}
\label{sec:project_evaluation-product_evaluation}



\section{Conclusion}
\label{sec:project_evaluation-conclusion}

%All in all, the group felt that the project worked out well. Luckily, there were no social issues, and both the customer and supervisor were friendly and helpful. Proper planning proved to be key, especially in a field in which the group had so little domain knowledge. Time management was the most difficult aspect of the whole project. It proved really hard to evaluate how much time tasks would take, and the group suffered from underestimating just how time-consuming some tasks were. That is definitely an aspect worth remembering for future work and development projects.

INSERT CONCLUSION ABOUT PRODUCT.

\clearpage