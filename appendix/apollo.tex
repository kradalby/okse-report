\chapter{Apache Apollo}
\label{appendix:apache-apollo}

\section{Introduction}
In this section the collected information and impressions of Apache Apollo is listed and analysed. As the research phase was a considerable large part of the project and was in a great degree focused on Apache Apollo, it is considered important.

The customer also indicated that they where very interested in all the research done on message brokers and existing solutions.

This appendix can be used as a standalone document, and may therefore repeat some information that also is mentioned in the main report for OKSE message broker.

\section{About Apollo}
Apache Apollo is a message broker system developed by the Apache foundation. Apollo is considered a "sub" project of Apache ActiveMQ\footnote{\url{http://activemq.apache.org}} and is by the developers own words, the next generation of ActiveMQ. Apollo is, as ActiveMQ a protocol agnostic broker what can freely transfer between the supported protocols. As of this writing, the supported protocol is MQTT, AMQP, STOMP, OpenWire and WebSockets. 

Apollo is developed under, as the name states, the Apache software foundation. Under the research phase, it was observed that most of the commits to the project was done by software developers from Red Hat\footnote{\url{http://www.redhat.com/en}}. The development team of OKSE considered this a positive thing, as they are probably committing to the project, in favor of their customers. 

\section{Implementation}

\subsection{Architecture}

\subsection{Scala}

\subsection{Points of entry}


\section{Evaluation}

\subsection{WSN support}

\clearpage
