\chapter{User Manual}

\section{About}

This chapter contains instructions for installing and using the OKSE Message Broker. The intended users are primarily researchers working with collaborative networking (and network based defence) (gir egentlig ikke mening å nevne). 

\section{Software Requirements}

OKSE Message Broker requires a Java Runtime Environment, or JRE\textsuperscript{TM}. It's developed on UNIX-systems like OS X 10.10 and Debian 8, as well as Windows 7. All operating systems with a JRE supports the application. Java SE Runtime Environment 8 (version 8u45) or newer is required. 

\section{Installation}

\subsection{Java Runtime Environment}

As previously stated, OKSE Message Broker requires a JRE 8. The broker is developed, and only tested, with this version. This software is available for free from Oracle.com\footnote{\url{http://www.oracle.com/technetwork/java/javase/downloads/jre8-downloads-2133155.html}}.

If you are using a UNIX-based system like Linux or OS X, you must remember to add Java to your path variable. Entering the following commands in a terminal should most likely work: [TODO: ADD FOR WINDOWS]

\begin{center}
\begin{tabular}{c}
\begin{lstlisting}[language=bash]
    # export JAVA_HOME=<PATH-TO-JRE>
    # export PATH=$JAVA_HOME:$PATH
\end{lstlisting}
\end{tabular}
\end{center}

\subsection{Download software}

Download the latest version of OKSE Message Broker from [URL HERE].
On UNIX-based systems, the file can be put into for instance \verb!$HOME/okse/!. On Windows, the file can be put into for instance \verb!C:\Program Files\okse\!. 

\subsection{Running software}

On Windows the broker can be started by double-clicking on the .jar-file. On UNIX-based systems and Windows alike, the broker could be started by running the command:

\begin{center}
\begin{tabular}{c}
\begin{lstlisting}[language=bash]
    # java -jar <PATH-TO-JAR-FILE>/<JAR-FILE>.jar
\end{lstlisting}
\end{tabular}
\end{center}

See section \ref{sec:inital-login} for more information regarding how to access the administration panel.

\section{Configuration}

After the initial start up of OKSE Message Broker, there is created a folder named \verb!config! in the same location as the .jar-file. This folder contains three configuration files: 

\begin{itemize}
\setlength{\itemsep}{0cm}%
\item okse.properties
\item log4j.properties
\item topicmapping.properties
\end{itemize}

\noindent See section \ref{sec:configuration-files} for detailed information about all possible settings.

\section{Inital login}
\label{sec:inital-login}

After installation and the first launch, the OKSE Admin Console can be accessed using a web-browser. By default, the admin console is accessible from \verb!localhost:8080!. Both the host and port can be changed in the okse.properties (see section \ref{sec:configuration_okse.properties}). To log in for the first time, use \verb!admin! and \verb!password! as respectively username and password. To increase the security, it is recommended that this is changed immediately after the inital login. 

\begin{itemize}
\setlength{\itemsep}{0cm}%
\item Default host and port for admin console: \verb!localhost:8080!
\item Default username and password: \verb!admin! and \verb!password!
\end{itemize}

\section{Admin Console}
The OKSE Admin Console (hereby denoted as OAC) provides you information about OKSE and it's operation enviroment. It also give you access to configure OKSE. The OAC have information sorted in 6 panes, the Main pane, the Topic pane, the Subscribers pane and the Configuration pane.

\subsection{The Main pane}
The Main pane is an overview of the status of the broker, system, the operating environment, and the interfaces the different protocol-servers are bound to. It is also possible to start and stop the protocol-servers from the interface section.

\subsection{The Topics pane}
The Topics pane contains all the registered topics in OKSE. For each topic it is possible to see how many clients are subscribing to the topic (---comment about xpath not being subscribers??---). It is also possible to delete all topics or a specific topic. The "stop refresh" button stops the automatic refreshing of the topic list. (-- noe om hvilke situasjoner stop knappen er tenkt til

\subsection{The Subscribers pane}
The subscribers pane shows all the subscribers (hosts). It also lists which protocol the subscriber are using, on which port, and if any filters are specified. One or all users can be deleted, denoted as unsubscribe in WSN and disconnect in AMQP. The "stop refresh" button stops the automatic refreshing of the subscription list.

\subsection{The Statistics pane}
The Statistics pane prvides a list of protocol-servers with statistics per protocol and total usage. "Sent" denotes the amount of messages sent on the different protocols. "Received" denotes messages received on current protocol. "Requests" are the total amount of requests on the protocol. (--skriv om hvordan det fungerer for både WSN og AMQP--). Bad request (---her---). Errors (---Her---). In the usage box you get the total stats for all protocols.

\subsection{The Configuration pane}
The Configuration pane have three sections, options, topic mapping and relays. In the option section you can change the auto update interval for refreshing the webpage. The "Use AMQP queues instead of pub/sub topics" changes the queue semantic.

In the topic mapping section one can manually map from one topic to another one. The mapping is an asynchronous action. If a mapping is made between topic A and B, messages sent to A will also be forwarded to topic B. Messages sent to topic B will not be forwarded to topic A.

\subsection{The Logs pane}
The main feature of the "Logs" pane is the large text area, providing all of the information printed from the application. Logging is separated into 4 levels, "ERROR", "WARN", "INFO" and "DEBUG". The "DEBUG" level shows the full set of messages printed during system runtime. "INFO" prints messages that are meant to show the progress and events happening in the application. The "WARN" level prints messages that might potentially lead to system failure. Finally, "ERROR" shows events which has caused a failure in the system. There are one button for each of the message types, allowing the user to filter out information. The "Lines" area, lets the user decide how many lines of information that are displayed. Lastly, the "Stop refresh" buttons stops new messages from showing up in the text-area.

\section{Sending messages}

\section{Configuration files}
\label{sec:configuration-files}
OKSE has 3 configurations files. The main configuration file is the okse.properties file. You also got log4j.properties, and topicmapping.properties.

\subsection{okse.properties}
\label{subsec:okse.properties}
 
 This is the main configuration file within the system. See below for a detailed list of all possible settings, and their description.

\begin{description}
\setlength{\itemsep}{0cm}%
  \item[sprint.application.name] \hfill \\
  Application name to be displayed in the Admin Console \hfill \\ Default: \verb!OKSE Message Broker!
  \item[server.port] \hfill \\
  Tells Jetty which port the Admin Console should use \hfill \\ Default: \verb!8080!
  \item[ADMIN\_PANEL\_HOST] \hfill \\
  The host the Admin Console should use \hfill \\ Default: \verb!0.0.0.0!
  \item[CACHE\_MESSAGES] \hfill \\
  Tells OKSE if it should cache messages \hfill \\ Default: \verb!true!
  \item[BROADCAST\_SYSTEM\_MESSAGES\_TO\_SUBSCRIBERS] \hfill \\
  Tells OKSE if system messages shoudl be broadcasted to all subscribers \hfill \\ Default: \verb!false!
  \item[ENABLE\_WSNU\_DEBUG\_OUTPUT] \hfill \\
  Tells OKSE if it should log WSNU \hfill \\ Default: \verb!false!
  \item[DEFAULT\_SUBSCRIPTION\_TERMINATION\_TIME] \hfill \\
  Tells OKSE what the subscription termination time should be \hfill \\ Default: \verb!15552000000!
  \item[DEFAULT\_PUBLISHER\_TERMINATION\_TIME] \hfill \\
  Tells OKSE what the publisher termination time should be \hfill \\ Default: \verb!15552000000!
  \item[TOPIC\_MAPPING] \hfill \\
  Tells OKSE the path to the topic mapping configuration file \hfill \\ Default: \verb!config/topicmapping.properties!
  \item[WSN\_HOST] \hfill \\
  Tells OKSE what host WSNotificanServer should listen to \hfill \\ Default: \verb!0.0.0.0!
  \item[WSN\_PORT] \hfill \\
  Tells OKSE what port WSNotificationServer should listen to \hfill \\ Default: \verb!61000!
  \item[WSN\_CONNECTION\_TIMEOUT] \hfill \\
  Tells OKSE what connection timeout to use with WS-Notification \hfill \\ Default: \verb!5!
  \item[WSN\_POOL\_SIZE] \hfill \\
   WSNotification http client thread pool used to queue outbound requests \hfill \\ Default: \verb!50!
   \item[WSN\_MESSAGE\_CONTENT\_ELEMENT\_NAME] \hfill \\
  Tells OKSE what name non-XML content should be wrapped in \hfill \\ Default: \verb!Content!
  \item[WSN\_USES\_NAT] \hfill \\
  Tells OKSE if it's hosted behind NAT/Port forwarded network \hfill \\ Default: \verb!false!
  \item[WSN\_WAN\_HOST] \hfill \\
  Tells OKSE what host it's behind when using NAT \hfill \\ Default: \verb!test.doman.com!
  \item[WSN\_WAN\_PORT] \hfill \\
  Tells OKSE what port it's behind when using NAT \hfill \\ Default: \verb!61000!
  \item[DUMMYPROTOCOL\_HOST] \hfill \\
  Tells OKSE what host DummyProtocolServer should listen to \hfill \\ Default: \verb!0.0.0.0!
  \item[DUMMYPROTOCOL\_PORT] \hfill \\
  Tells OKSE what port DummyProtocolServer should listen to \hfill \\ Default: \verb!61001!
  \item[AMQP\_HOST] \hfill \\
  Tells OKSE what host AMQPProtocolServer should listen to \hfill \\ Default: \verb!0.0.0.0!
  \item[AMQP\_PORT] \hfill \\
  Tells OKSE what port AMQPProtocolServer should listen to \hfill \\ Default: \verb!5672!
  \item[AMQP\_USE\_QUEUE] \hfill \\
  Tells OKSE if AMQP should use the non-standard topic implementation \hfill \\ Default: \verb!true!
  \item[AMQP\_USE\_SASL] \hfill \\
  Tells OKSE if AMQP should use SASL \hfill \\ Default: \verb!true!
  \item[spring.resources.cache-period] \hfill \\
  Tells Spring what cache-period to set on HTTP-requests \hfill \\ Default: \verb!1!
  \item[spring.thymeleaf.suffix] \hfill \\
  Tells Spring what all Thymeleaf templates are suffixed with \hfill \\ Default: \verb!.html! 
   \item[spring.thymeleaf.mode] \hfill \\
  Tells Spring what type all Thymeleaf templates are \hfill \\ Default: \verb!HTML5!
   \item[spring.thymeleaf.encoding] \hfill \\
  Tells Spring what encoding to use on all Thymeleaf templates \hfill \\ Default: \verb!UTF-8!
   \item[spring.thymeleaf.content-type] \hfill \\
  Tells Spring what content-type all Thymeleaf templates should be returned with \hfill \\ Default: \verb!text/html! 
\end{description}
  
 \subsection{log4j.properties}
 \label{subsec:log4j.properties}
 
This is the configuration file for all the log files available for the system. See below for a detailed list of all possible settings, and their description.

\begin{description}

\setlength{\itemsep}{0cm}%
  \item[log] \hfill \\
  Default folder location for log files, relative to .jar file \hfill \\ Default: \verb!logs!
  \item[pattern] \hfill \\
  Default pattern to print log output in \hfill \\ Default: \verb!%d{yyyy-MM-dd - HH:mm:ss.SSS} [%p] (%t) %c: - %m%n!
   \item[maxLogFileSize] \hfill \\
  Max log file size, before log rotate \hfill \\ Default: \verb!5MB!
   \item[numberOfBackups] \hfill \\
  Max number of log files, before it purges old files \hfill \\ Default: \verb!10!
   \item[log4j.logger.no.ntnu.okse] \hfill \\
  Default log level for okse log messages \hfill \\ Default: \verb!DEBUG, OKSE!
  \item[log4j.logger.org.apache.qpid] \hfill \\
  Default log level for qpid log messages \hfill \\ Default: \verb!DEBUG, QPID!
  \item[log4j.logger.org.eclipse.jetty] \hfill \\
  Default log level for Jetty log messages \hfill \\ Default: \verb!INFO, JETTY!
  \item[log4j.logger.org.springframework] \hfill \\
  Default log level for Spring log messages \hfill \\ Default: \verb!INFO, SPRING!
  
  \item[log4j.appender.OKSE] \hfill \\
  Default appender to use for OKSE logs \hfill \\ Default: \verb!org.apache.log4j.RollingFileAppender!
  \item[log4j.appender.OKSE.File] \hfill \\
  Default log file to use for OKSE logs \hfill \\ Default: \verb!${log}/okse.log!
   \item[log4j.appender.OKSE.MaxFileSize] \hfill \\
  Default log file size to use for OKSE logs \hfill \\ Default: \verb!${maxLogFileSize}!
   \item[log4j.appender.OKSE.MaxBackupIndex] \hfill \\
  Number of backups for OKSE logs \hfill \\ Default: \verb!${numberOfBackups}!
   \item[log4j.appender.OKSE.layout] \hfill \\
  Pattern engine for OKSE logs \hfill \\ Default: \verb!org.apache.log4j.PatternLayout!
   \item[log4j.appender.OKSE.layout.conversionPattern] \hfill \\
  Default pattern to use for OKSE logs \hfill \\ Default: \verb!${Pattern}!

    \item[log4j.appender.SPRING] \hfill \\
  Default appender to use for SPRING logs \hfill \\ Default: \verb!org.apache.log4j.RollingFileAppender!
  \item[log4j.appender.SPRING.File] \hfill \\
  Default log file to use for SPRING logs \hfill \\ Default: \verb!${log}/spring.log!
   \item[log4j.appender.SPRING.MaxFileSize] \hfill \\
  Default log file size to use for SPRING logs \hfill \\ Default: \verb!${maxLogFileSize}!
   \item[log4j.appender.SPRING.MaxBackupIndex] \hfill \\
  Number of backups for SPRING logs \hfill \\ Default: \verb!${numberOfBackups}!
   \item[log4j.appender.SPRING.layout] \hfill \\
  Pattern engine for SPRING logs \hfill \\ Default: \verb!org.apache.log4j.PatternLayout!
   \item[log4j.appender.SPRING.layout.conversionPattern] \hfill \\
  Default pattern to use for SPRING logs \hfill \\ Default: \verb!${Pattern}!

  \item[log4j.appender.JETTY] \hfill \\
  Default appender to use for JETTY logs \hfill \\ Default: \verb!org.apache.log4j.RollingFileAppender!
  \item[log4j.appender.JETTY.File] \hfill \\
  Default log file to use for JETTY logs \hfill \\ Default: \verb!${log}/okse.log!
   \item[log4j.appender.JETTY.MaxFileSize] \hfill \\
  Default log file size to use for JETTY logs \hfill \\ Default: \verb!${maxLogFileSize}!
   \item[log4j.appender.JETTY.MaxBackupIndex] \hfill \\
  Number of backups for JETTY logs \hfill \\ Default: \verb!${numberOfBackups}!
   \item[log4j.appender.JETTY.layout] \hfill \\
  Pattern engine for JETTY logs \hfill \\ Default: \verb!org.apache.log4j.PatternLayout!
   \item[log4j.appender.JETTY.layout.conversionPattern] \hfill \\
  Default pattern to use for JETTY logs \hfill \\ Default: \verb!${Pattern}!
  
    \item[log4j.appender.QPID] \hfill \\
  Default appender to use for QPID logs \hfill \\ Default: \verb!org.apache.log4j.RollingFileAppender!
  \item[log4j.appender.QPID.File] \hfill \\
  Default log file to use for QPID logs \hfill \\ Default: \verb!${log}/qpid.log!
   \item[log4j.appender.QPID.MaxFileSize] \hfill \\
  Default log file size to use for QPID logs \hfill \\ Default: \verb!${maxLogFileSize}!
   \item[log4j.appender.QPID.MaxBackupIndex] \hfill \\
  Number of backups for QPID logs \hfill \\ Default: \verb!${numberOfBackups}!
   \item[log4j.appender.QPID.layout] \hfill \\
  Pattern engine for QPID logs \hfill \\ Default: \verb!org.apache.log4j.PatternLayout!
   \item[log4j.appender.QPID.layout.conversionPattern] \hfill \\
  Default pattern to use for QPID logs \hfill \\ Default: \verb!${Pattern}!

 \item[log4j.appender.stdout] \hfill \\
  Default appender to use for console output \hfill \\ Default: \verb!org.apache.log4j.ConsoleAppender!
   \item[log4j.appender.stdout.Target] \hfill \\
  Default target to use for console output \hfill \\ Default: \verb!System.out!
    \item[log4j.appender.stdout.layout] \hfill \\
  Pattern engine for console logs \hfill \\ Default: \verb!org.apache.log4j.PatternLayout!
   \item[log4j.appender.stdout.layout.conversionPattern] \hfill \\
  Default pattern to use for console logs \hfill \\ Default: \verb!${Pattern}!
  
 \end{description}
 
 \subsection{topicmapping.properties}
 \label{subsec:topicmapping.properties}
 
 This is a configuration file for adding predefined topic mappings to be available when the broker boots. By default, there are no topic mapping listed. To add predefined topic mappings, use the following format: 
 
 \begin{center}
\begin{tabular}{c}
\begin{lstlisting}[]
    from = to
    no/ffi = nato/hq/info
\end{lstlisting}
\end{tabular}
\end{center}

\section{Troubleshooting}

If you encounter any problems with the software, check the following points before you contact the development team: 

\begin{itemize}
\setlength{\itemsep}{0cm}%
\item Do you have the correct JRE? For OKSE Message Broker to run properly, you need version 8 or newer. Other versions than this may cause problems, or not work at all.
\item Network connectivity. Verify that your Internet connection is working and that the ports are open. Also ensure that you use port forwarding if you don't have a public IP.
\end{itemize}

\clearpage