\clearpage
\pagenumbering{roman} 				
\setcounter{page}{1}

\pagestyle{fancy}
\fancyhf{}
\renewcommand{\chaptermark}[1]{\markboth{\chaptername\ \thechapter.\ #1}{}}
\renewcommand{\sectionmark}[1]{\markright{\thesection\ #1}}
\renewcommand{\headrulewidth}{0.1ex}
\renewcommand{\footrulewidth}{0.1ex}
\fancyfoot[LE,RO]{\thepage}
\fancypagestyle{plain}{\fancyhf{}\fancyfoot[LE,RO]{\thepage}\renewcommand{\headrulewidth}{0ex}}

\section*{\Huge Summary}
\addcontentsline{toc}{chapter}{Summary}	
$\\[0.5cm]$

\noindent This report describes the work done in the course IT2901 - Prosjektarbeid i informatikk II. The customer was the Norwegian Defence Research Establishment(hereby denoted as FFI) which is a governmental organization responsible for research and development for the Norwegian Armed Forces. The assignment was to make an application that translates between different publish/subscribe protocols used in the Norwegian Armed Forces and the North Atlantic Treaty Organisation (hereby denoted as NATO). NATO uses the Web Services Notification protocol (hereby denoted as WSN) as their standard, while the Norwegian Armed Forces also uses other protocols like Advanced Messaging Queuing Protocol (hereby denoted as AMQP), Message Queue Telemetry Transport (hereby denoted as MQTT) and ZeroMQ. FFI and The Norwegian Armed Forces needed such a broker in order to participate in NATO's federated mission network (hereby denoted as FMN). This report mainly focuses on the development process. However, a thorough description of the product is also included.

\clearpage