%===================================== CHAP 3 =================================

\chapter{Requirements engineering}

The requirements engineering process is a negotiation between the customer and the development team. This process will go on throughout the duration of the project. Thus, the requirements listed here is a result of early interaction with the customer, and is in no way a finalized version of the requirements.

Furthermore, requirements can be seperated into two categories. A functional requirement describes a function of the system or one of it's components. On the other hand, a non-functional requirement can be viewed with regards to the operation of the system. That is, rather than listing specific functions of the system, it describes the performance or operation of the system as a whole. An example might be that the administration interface needs to show data no older than 30 seconds.

\section{Functional requirements}

The requirements listed below, are in descending order of importance. In the current phase of the process, these are high-level requirements, which means that there's little to no details about how it shall be implemented.

\subsection{FR1 - WS-Notification}

The system must support the Web Services Notification (WSN). It needs to be able to translate both to and from the standard.

\subsection{FR2 - Protocol independency}

The system must be able to translate between protocols independent of their type, more precisely it must not only support to/from WSNotification, but also between protocols like AMQP and MQTT, without having to go through WSNotification.

\subsection{FR3 - Administrator interface}

The application must have a graphical administration interface. The administrator interface must expose configuration of mappings between supported protocols, publisher and subscriber database listings, as well as topics and content filters.

\subsection{FR4 - AMQP}

The system must support the messaging protocol AMQP. It needs to be able to translate both to and from the protocol.

\subsection{FR5 - MQTT}

The system must support the MQTT protocol. It needs to be able to translate both to and from the protocol.

\section{Non-functional requirements}

\subsection{NFR1 - Documentation}

All parts of the application must be documented according to the language specific standards, preferrably in english.

\subsection{NFR2 - Module based software}

The different parts of the system should be modularized, to allow the addition of more protocols at a later point in time. Additionally, to allow upgrading og further developing different parts of the system individually.


\cleardoublepage