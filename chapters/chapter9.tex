%===================================== CHAP 9 =================================

\chapter{Further development}

\section{Protocols}
A variety of proposals for other protocols that may be implemented are listed below. Most of them are based on the original wishes from FFI. This chapter is meant to explain the parts of the system that can or should be improved. 

\subsection{AMQP changes and more versions}
The current implementation of AMQP will, while queue behaviour is configured, deliver the message to a random connected recipient connected to the current queue. This functionality could possibly be extended to supporting multiple algorithms for choosing. An example of this kind of algorithm can be Round-robin.

It can also be useful, should the need arise, to implement older version of the AMQP protocol. As mentioned in section \ref{subsec:prestudies-amqp-amqp_version}, the older versions of the AMQP protocol are radically different, but still widely implemented.

\subsubsection{Authentication}
The system currently has support for SASL \cite{sasl}, but the implementation will only accept ANONYMOUS as the authentication criteria. Further development should implement a solution with actual user access. 

\subsection{MQTT}
As MQTT was one of the original planned protocols, it makes sense to view this as the next step when it comes to extending OKSE with new functionality. MQTT is a pure publish/subscribe pattern and the integration with OKSE should be fairly compatible. 
During the research phase, when the group did research on different libraries for each protocol, the most promising Java implementation discovered was Moquette MQTT \cite{moquette-mqtt}. No deep research was performed, but the fact that it had a lot of traction, and that it was backed by the Eclipsed foundation gave a positive impression.

\subsection{ZeroMQ}
The next protocol that should be looked into is ZeroMQ, it was also on the wish list from FFI, but had a lower priority. ZeroMQ is by some considered to be the biggest competitor to AMQP. It was developed by iMatix Corporation \cite{imatix}, one of the original authors of AMQP. If ZeroMQ is considered for implementation, the AMQP implementation should be analyzed thoroughly and many of the same concepts should apply. Some research should be done on the benefits of using JeroMQ \cite{jero-mq}, which is a pure java implementation. This implementation stands in contrast to the C implementation with Java bindings. Both versions are developed by the ZeroMQ team, and are considered "drop-in" replacements of one another.

\section{Web Technologies}
Even though the administration interface proved to be stable during testing, some aspects of the architecture may be subject to change in the future. This section is meant to explain what parts of the administration interface that can or should be improved first. 

\subsection{WebSockets}
As of now, the administration panel hasn't been tested with more than six simultaneous users. No issues were discovered, and the server load were minimal. But for future reference it would make sense to replace the auto update AJAX with WebSockets to reduce the server load. It would also guarantee that the users see the most up to date information. See section \ref{subsec:Web_technologies} for more detailed information regarding the use of WebSockets contra AJAX.

\clearpage
