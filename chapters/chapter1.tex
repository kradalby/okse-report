%===================================== CHAP 1 =================================

\chapter{Introduction}
\label{ch:introduction}

\section{The course}
\label{sec:introduction-the_course}

The goal of the course IT2901 is to give students experience in working on a project with a customer. The students are to work in groups to develop a product related to information technology, with guidance from a supervisor. This project includes all parts of the software development process up to, but excluding, the maintenance/evolution phase.

\section{The group}
\label{sec:introduction-the_group}

Our group consist of six members; Fredrik Christoffer Berg, Kristoffer Andreas Breiland Dalby, Håkon Ødegård Løvdal, Aleksander Skraastad, Fredrik Borgen Tørnvall and Trond Walleraunet. All of the members started a bachelors degree in computer science at NTNU in the fall of 2012. Over the course of the study, the members have gained experience in programming, as well as methodologies in software development.

\subsubsection{Fredrik Christoffer Berg}

Berg has prior experience in programming from courses taught at NTNU. Additionally, he has some experience with writing papers, both from school and in other areas. He is also interested in, and has a fairly good understanding, of different development methodologies.

\subsubsection{Kristoffer Andreas Breiland Dalby}

Dalby has prior experience with backend development and server administration. He also has experience with the programming languages Java, Python and JavaScript.

\subsubsection{Håkon Ødegård Løvdal}

Løvdal has experience with Java, Python and JavaScript from courses taught at NTNU. He also has knowledge of the Python framework Django, HTML and CSS through personal projects. Furthermore, Løvdal has gained experience with human-computer interaction from being a student assistant in the course TDT4180 at NTNU during the lifetime of this project. 

\subsubsection{Aleksander Skraastad}

Skraastad has experience with Java, Python, HTML, CSS and JavaScript, as well as some experience with SQL databases. He had been involved in a few medium scale programming projects before, both commercially, and through volunteering and private projects.

\subsubsection{Fredrik Borgen Tørnvall}

Tørnvall has past experience from courses at NTNU. Programming experience includes Java, Python, C\# and JavaScript. Additionally, he has past work experience working in small to medium sized teams. Finally, he has experience with Android development. 

\subsubsection{Trond Walleraunet}

Walleraunet has experience with Java, Python, JavaScript, HTML and CSS through courses taught at NTNU. Additionally, he has work experience in server and network management. Finally, he has experience with project management working in small to medium sized teams.


\section{The customer}
\label{sec:introduction-the_customer}

The customer that we have been cooperating with on this project is the Norwegian Defence Research Establishment (hereby denoted FFI). FFI is a governmental organization responsible for research and development for the Norwegian Armed Forces. In addition, the organization is involved in the long term planning of the armed forces, as well as participating in other non-military research projects. The customer was located at Kjeller in Oslo. Thus, communication was mainly performed with the help of video calls.

\section{Project Description}
\label{sec:introduction-project_description}

FFI was in need of an application aimed for translation between various publish/subscribe communication protocols. The application would include both protocols used internally in the Norwegian Armed Forces and the North Atlantic Treaty Organisation (hereby denoted NATO). The standard protocol for this type of communication is the Web Services Notification protocol (hereby denoted WSN) \cite{wsn-complete}, which was the main focus of FFI.

At the time, FFI did not have such an application. They did however possess an implementation of the WSN protocol, developed during a project at NTNU in 2014. The group's goal was to develop software that could provide efficient communication between vehicles, stations, personnel, unmanned aerial vehicles (UAVs) and international NATO forces. Additionally, it would be a challenge develop, as the criteria given in the requirements could present the need for a deeper understanding of various publish/subscribe technologies. This project could yield an important application that would allow quick adaptation of new devices and tools that use other protocols than the NATO standard.

The task of this project was to implement a multi-protocol publish/subscribe brokering solution. Although there exists several different publish/subscribe frameworks and protocols, our goal was to make one solution which covered, and was able to translate, between several of them. The primary function however, was for the solution to support WSN, Advanced Message Queuing Protocol (hereby denoted AMQP) \cite{amqp} and Message Queuing Telemetry Transport (hereby denoted MQTT) \cite{mqtt}. ZeroMQ \cite{zero-mq} and Data Distribution Service (hereby denoted DDS) \cite{dds} were also interesting protocols, but with lower priority in this project. The customer wanted the application to translate to and from all the implemented protocols, regardless of their type. The application also had to include a graphical interface for administration. Note that clients for sending and receiving messages were not part of the project.

Finally, there were no security requirements defined. The group was assured that security would be implemented in the Internet layer (e.g IPSec) or link layer (e.g L2TP).


